\documentclass{article}
\usepackage{amsmath}
\title{Module 3 Discussion}
\author{Pedro G\'{o}mez Mart\'{i}n}
\date{\today}
\begin{document}
\maketitle
There are geometric and arithmetic series and sequences the letter $n$ denotes the position of the term in the sequence, arithmetic can be expressed as $x_{n}=an+b$ where $a$ is the initial value and $b$ is the constant added each time, this can also be expressed as $x_{n}=x_{n-1}+b$ where $x_{0}=a$
and geometric as $x_{n}=ar^{n-1}$ where $a$ is the initial value and $r$ the common ratio. Also, they can be represented as $x_{n}=rx_{n-1}$ where $x_{0}=a$
\\
Sequences are ordered lists that follow a rule. Such as $a_{n}=a+n$ \\
Sequences can be finite or infine, here is an example
\begin{equation*}
\left\lbrace a_{n}\right\rbrace ^{\infty}_{n=3} \ \ a_{n}=\left( \frac{1}{2} \right)^{n-1}
\end{equation*}
Series are expressed with the upper-case greek letter ``sigma'' $\Sigma$ and are the sum of a sequence, here is an example of an infinite serie:
\begin{equation*}
\sum_{n=1}^{\infty}{\left( \frac{1}{2} \right)^{n-1}}
\end{equation*}
\end{document}