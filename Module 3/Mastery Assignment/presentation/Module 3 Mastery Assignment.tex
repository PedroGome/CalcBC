\documentclass[10pt]{beamer}
\usetheme[
%%% option passed to the outer theme
%    progressstyle=fixedCircCnt,   % fixedCircCnt, movingCircCnt (moving is deault)
  ]{Feather}
  
% If you want to change the colors of the various elements in the theme, edit and uncomment the following lines

% Change the bar colors:
%\setbeamercolor{Feather}{fg=red!20,bg=red}

% Change the color of the structural elements:
%\setbeamercolor{structure}{fg=red}

% Change the frame title text color:
%\setbeamercolor{frametitle}{fg=blue}

% Change the normal text color background:
%\setbeamercolor{normal text}{fg=black,bg=gray!10}

%-------------------------------------------------------
% INCLUDE PACKAGES
%-------------------------------------------------------

\usepackage[utf8]{inputenc}
\usepackage[american]{babel}
\usepackage[T1]{fontenc}
\usepackage{helvet}
\usepackage{mathtools}
\usepackage{multicol}
\usepackage{setspace}
\usepackage{fancyhdr}
\usepackage{tabularx}
\usepackage{graphicx}

%-------------------------------------------------------
% DEFFINING AND REDEFINING COMMANDS
%-------------------------------------------------------

% colored hyperlinks
\newcommand{\chref}[2]{
  \href{#1}{{\usebeamercolor[bg]{Feather}#2}}
}

%-------------------------------------------------------
% INFORMATION IN THE TITLE PAGE
%-------------------------------------------------------

\title[] % [] is optional - is placed on the bottom of the sidebar on every slide
{ % is placed on the title page
      \textbf{Module 3 Mastery Assignment}
}

\subtitle[]
{
      \textbf{ }
}

\author[Pedro G\'{o}mez Mart\'{i}n]
{      Pedro G\'{o}mez Mart\'{i}n}

\institute[]
{

  
  %there must be an empty line above this line - otherwise some unwanted space is added between the university and the country (I do not know why;( )
}

\date{\today}

%-------------------------------------------------------
% THE BODY OF THE PRESENTATION
%-------------------------------------------------------

\begin{document}

%-------------------------------------------------------
% THE TITLEPAGE
%-------------------------------------------------------

{\1% % this is the name of the PDF file for the background
\begin{frame}[plain,noframenumbering] % the plain option removes the header from the title page, noframenumbering removes the numbering of this frame only
  \titlepage % call the title page information from above
\end{frame}}


\begin{frame}{Content}{}
\tableofcontents
\end{frame}

%-------------------------------------------------------
\section{Taylor Polynomials}
%-------------------------------------------------------
\subsection{Intro}
\begin{frame}{Taylor Polynomials}{Intro}
%-------------------------------------------------------
We want to approximate the function $f(x)$ that satisfies the following conditions:
\begin{itemize}
	\item Is a real or a complex-value function.
	\item It is infinitely differentiable at $c$
\end{itemize}

\end{frame}

\subsection{Intuitive Derivation}
\begin{frame}{Taylor Polynomials}{Intuitive Derivation}
%-------------------------------------------------------
\begin{align*}
	P\left(0\right)=&f\left(0\right)\\
	P\left(x\right)=&f\left(0\right)
\end{align*}
	
\end{frame}

\begin{frame}{Taylor Polynomials}{Intuitive Derivation}
%-------------------------------------------------------
	\begin{align*}
	P^{'}\left(0\right)=&f^{'}\left(0\right)\\
	P\left(x\right)=&f\left(0\right)+f^{'}\left(0\right)x
	\end{align*}
	
\end{frame}

\begin{frame}{Taylor Polynomials}{Intuitive Derivation}
%-------------------------------------------------------
	\begin{align*}
	P^{''}\left(0\right)=&f^{''}\left(0\right)\\
	P\left(x\right)=&f\left(0\right)+f^{'}\left(0\right)x+\frac{1}{2}f^{''}\left(0\right)x^2
	\end{align*}
	
\end{frame}

\begin{frame}{Taylor Polynomials}{Intuitive Derivation}
	%-------------------------------------------------------
	\begin{align*}
	P^{'''}\left(0\right)=&f^{'''}\left(0\right)\\
	P\left(x\right)=&f\left(0\right)+f^{'}\left(0\right)x+\frac{1}{2}f^{''}\left(0\right)x^2+\frac{1}{2\cdot 3}f^{'''}\left(c\right)x^3
	\end{align*}
	
\end{frame}

\begin{frame}{Taylor Polynomials}{Intuitive Derivation}
	%-------------------------------------------------------
	\begin{align*}
	P^{\left(4\right) }\left(0\right)=&f^{\left(4\right)}\left(0\right)\\
	P\left(x\right)=&f\left(0\right)+f^{'}\left(0\right)x+\frac{1}{2}f^{''}\left(0\right)x^2+\frac{1}{2\cdot 3}f^{'''}\left(c\right)x^3+\frac{1}{2\cdot 3\cdot 4}f^{(4)}\left(0\right)x^4
	\end{align*}
	
\end{frame}

\begin{frame}{Taylor Polynomials}{Intuitive Derivation}
%-------------------------------------------------------
	\begin{align*}
	P^{\left(4\right) }\left(0\right)=&f^{\left(4\right)}\left(0\right)\\
	P\left(x\right)=&f\left(0\right)+f^{'}\left(0\right)x+\frac{1}{2}f^{''}\left(0\right)x^2+\frac{1}{2\cdot 3}f^{'''}\left(c\right)x^3+\frac{1}{2\cdot 3\cdot 4}f^{(4)}\left(0\right)x^4\\
	&+ \dots +\frac{1}{n!}f^{(n)}\left(0\right)x^n
	\end{align*}
	
\end{frame}

\begin{frame}{Taylor Polynomials}{Intuitive Derivation}
%-------------------------------------------------------
	\begin{align*}
	\sum\limits_{n=0}^{\infty}\frac{f^{(n)}\left(0\right)}{n!}\cdot x^n
	\end{align*}
	
\end{frame}

\begin{frame}{Taylor Polynomials}{Intuitive Derivation}
%-------------------------------------------------------
	\begin{align*}
		P{\left(x\right)}=&f\left(c\right)\ \longrightarrow\ P(c)=f(c)\\
		P\left(x\right) = &f\left(c\right) + f^{'}\left(c\right)(x-c) + \frac{1}{2}f^{''}\left(c\right)(x-c)^2 + \frac{1}{2\cdot 3}f^{'''}\left(c\right)(x-c)^3\\
		&+ \frac{1}{2\cdot 3\cdot 4}f^{(4)}\left(c\right)(x-c)^4 + \dots +\frac{1}{n!}f^{(n)}\left(c\right)(x-c)^n\\
		\ &\\
		P(c)=&f(c)+f^{'}(c)(c-c)+\cdots+\frac{f^{(n)}(c)}{n!}(x-c)^{n}
	\end{align*}
	
\end{frame}

\subsection{Example}
\begin{frame}{Taylor Polynomials}{Example}
%-------------------------------------------------------
	Approximating $\sin (x)$\\
	\begin{flalign*}
		&f(x)&=&\sin(x)&&\\
		&f^{'}(x)&=&\cos (x)&&\\
		&f^{''}(x)&=&-\sin (x)&&\\
		&f^{(3)}(x)&=&-\cos (x)&&
	\end{flalign*}
	
\end{frame}

\begin{frame}{Taylor Polynomials}{Example}
%-------------------------------------------------------
	\begin{flalign*}
	&f(0)&=&\sin(0)=0&&\\
	&f^{'}(0)&=&\cos (0)= 1&&\\
	&f^{''}(0)&=&-\sin (0)= 0&&\\
	&f^{(3)}(0)&=&-\cos (0)= -1&&
	\end{flalign*}
	
\end{frame}

\begin{frame}{Taylor Polynomials}{Example}
%-------------------------------------------------------
	Since it becomes cyclic, we can 
	\begin{flalign*}
		&\sum\limits_{n=0}^{\infty}\frac{f^{(n)}\left(0\right)}{n!}\left(x\right)^{n}&&\\
		&\frac{0}{0!}{x^0}+\frac{1}{1!}x^1+\frac{0}{2!}x^2+\frac{-1}{3!}x^3+\frac{0}{4!}x^4+\frac{1}{5!}x^5+\frac{0}{6!}x^6+\frac{-1}{7!}x^7&&\\
		&x-\left(3!\right)^{-1}x^3+\left(5!\right)^{-1}x^5-\left(7!\right)^{-1}x^7&&
	\end{flalign*}
	A clear pattern begins to form, we can condense it into:
	\begin{align*}
		\sum\limits_{n=1}^{\infty}\left(-1\right)^{\left(n+1\right)}\frac{x^{\left(2n-1\right)}}{\left(2n-1\right)!}
	\end{align*}
	
\end{frame}

%-------------------------------------------------------
\section{License}
%-------------------------------------------------------
\begin{frame}{License}{GNU GPLv3}
%-------------------------------------------------------
	\begin{block}{GNU GPLv3}
		This entire presentation is under the GNU General Public License v3\\
		\begin{figure}
			\centering
			\def\svgwidth{\columnwidth}
			\input{image.pdf_tex}
		\end{figure}
	\end{block}
	
\end{frame}

\end{document}
