\documentclass{article}
\usepackage{mathtools}
\usepackage{multicol}
\usepackage{setspace}
\usepackage[utf8]{inputenc}
\usepackage{fancyhdr}
\usepackage{tabularx}
\usepackage[american]{babel}
\usepackage{enumerate}


\usepackage[letterpaper,left=3cm,right=2cm,top=2.5cm,bottom=1.5cm]{geometry}
\onehalfspacing

\author{Pedro G\'{o}mez Mart\'{i}n}
\date{\today}
\title{Normal Distribution Derivation}

\begin{document}
\maketitle

\section{Intuition}

First, to derive this incredibly useful formula, we start by the intuition that
it is symmetric and that it is similar to the derivative of the logistic
equation, with that intuition, we can come up with the following differential
equation:
\begin{equation}
  \label{eq:Differential equation intuition}
  \frac{df}{dx}=-k(x-\mu)f(x)
\end{equation}
Where $k$ is the constant that defines the rate at which it decreases, $(x-\mu)$
describes the center, and $x$ describes the rate at which the frequencies
fall off proportionally to the distance of the score from the mean, and $f(x)$
to the frequencies themselves.

\section{Solution to the Intuition}
With a differential equation, we can now find a solution:
\begin{align}
  \frac{df}{f}&=-k(x-\mu)dx\\
  \int\frac{1}{f}df&=-k\int (x-\mu)dx\\
  \ln (f)&=\left[ \frac{-k(x-\mu)^2}{2} +C \right]\\
  f&=e^{-k\frac{\left(x-\mu\right)^2}{2}+C}\\
  f&=e^{-k\frac{\left(x-\mu\right)^2}{2}}e^C\\
  f&=Ce^{-k\frac{\left(x-\mu\right)^2}{2}}
\end{align}

\end{document}