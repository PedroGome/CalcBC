\documentclass{article}
\usepackage{mathtools}
\usepackage{multicol}
\usepackage{setspace}
\usepackage[utf8]{inputenc}
\usepackage{fancyhdr}
\usepackage{tabularx}
\usepackage[american]{babel}
\usepackage{enumerate}


\usepackage[letterpaper,left=3cm,right=2cm,top=2.5cm,bottom=1.5cm]{geometry}
\onehalfspacing

\author{Pedro G\'{o}mez Mart\'{i}n}
\date{\today}
\title{Normal Distribution Derivation}

\begin{document}
\maketitle

\section{Intuition}

First, to derive this incredibly useful formula, we start by the intuition that
it is symmetric and that it is similar to the derivative of the logistic
equation, with that intuition, we can come up with the following differential
equation:
\begin{equation}
  \label{eq:Differential equation intuition}
  \frac{df}{dx}=-k(x-\mu)f(x)
\end{equation}
Where $k$ is the constant that defines the rate at which it decreases, $(x-\mu)$
describes the center, and $x$ describes the rate at which the frequencies
fall off proportionally to the distance of the score from the mean, and $f(x)$
to the frequencies themselves.

\section{Solution to the Intuition}
With a differential equation, we can now find a solution:
\begin{align}
  \frac{df}{f}&=-k(x-\mu)dx\\
  \int\frac{1}{f}df&=-k\int (x-\mu)dx\\
  \ln (f)&=\left[ \frac{-k(x-\mu)^2}{2} +C \right]\\
  f&=e^{-k\frac{\left(x-\mu\right)^2}{2}+C}\\
  f&=e^{-k\frac{\left(x-\mu\right)^2}{2}}e^C\\
  f&=e^{-k\frac{\left(x-\mu\right)^2}{2}}C
\end{align}
\begin{equation*}
  \label{eq:General Solution}
  f(x)=Ce^{-k\frac{\left(x-\mu\right)^2}{2}}
\end{equation*}

\section{Specific Solution}
\subsection{Solve for C}
Now we know a general solution to the differential equation, but that by itself
is not helpful when trying to find the probability of an event that follows a
normal distribution.

To find this specific solution, we start by taking the definite integral from
$-\infty$ to $\infty$ since we know that the area under the curve is $1$.

\begin{equation}
  C\int\limits_{-\infty}^{\infty}e^{-\frac{k}{2}\left(x-\mu\right)^2}\ dx=1
\end{equation}

\subsubsection{U-Substitution}

To integrate this expression, u-substitution is really useful:

\begin{multicols}{3}
  \begin{equation}
    u=\sqrt{\frac{k}{2}}(x-\mu)
  \end{equation}
  \vfill\columnbreak
  \begin{equation}
    \frac{k}{2}(x-\mu^2)=u^2
  \end{equation}
  \vfill\columnbreak
  \begin{equation}
    du=\sqrt{\frac{k}{2}}dx\Longrightarrow dx=\sqrt{\frac{2}{k}}du
  \end{equation}
\end{multicols}
Therefore:
\begin{equation}
  C\int\limits_{-\infty}^{\infty}e^{-\frac{k}{2}(x-\mu)^2}dx=C\sqrt{\frac{2}{k}} \int\limits_{-\infty}^{\infty}e^{-u^2}du=1
\end{equation}

\subsubsection{Preparation for Polar Coordinates}

If we square the expression we obtain the following:
\begin{align}
  \left( C\sqrt{\frac{2}{k}}\int\limits_{-\infty}^{\infty}e^{-u^2}du \right)^2&=1^2\\
  \frac{2C^2}{k}\left( \int\limits_{-\infty}^{\infty}e^{-x^2}dx \right)  \left(\int\limits_{-\infty}^{\infty}e^{-y^2}dy \right)&=1
\end{align}
When the two integrals are separated, we can use different variables for each
one. Then, by Fubini's Theorem, we obtain the following:

\begin{align}
  \frac{2C^2}{k}\int\limits_{-\infty}^{\infty}\int\limits_{-\infty}^{\infty}e^{-\left[ x^2+y^2 \right]} dx\ dy = 1
\end{align}

\subsubsection{Translate to Polar Coordinates}

\begin{align}
  \frac{2C^2}{k}\int\limits_{0}^{2\pi}\int\limits_{0}^{\infty}re^{-r^2} dr\ d\theta = 1
\end{align}

\subsubsection{U-Substitution and Solving}

\begin{multicols}{2}
  $v = -r^2$
  \vfill\columnbreak
  $dv = -2r\ dr$
\end{multicols}

\begin{align}
  \frac{C^2}{k}\int\limits_{0}^{2\pi}\int\limits_{0}^{\infty}-e^{-v}dv\ d\theta &= 1\\
  \frac{C^2}{k}\int\limits_{0}^{2\pi}1\ d\theta &= 1\\
  \frac{C^2}{k}2\pi &= 1\\
  C^2 &= \frac{k}{2\pi}\\
  C &= \sqrt{\frac{k}{2\pi}}
\end{align}

\subsection{Solving for k}

\end{document}