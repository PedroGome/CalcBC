\documentclass{article}
\usepackage{mathtools}
\usepackage{amssymb}
\usepackage{multicol}
\usepackage{setspace}
\usepackage[utf8]{inputenc}
\usepackage{fancyhdr}
\usepackage{tabularx}
\usepackage[american]{babel}
\usepackage{lipsum}
\usepackage{enumitem}

\usepackage[letterpaper,left=3cm,right=2cm,top=2.5cm,bottom=1.5cm]{geometry}
\onehalfspacing

\author{Pedro G\'{o}mez Mart\'{i}n}
\date{\today}
\title{Module 4 Review Assignment}

\begin{document}
\maketitle

\begin{flalign*}
  x=e^{t} \ \&\ y=\cos{t}
\end{flalign*}

  \begin{enumerate}
  \item\textbf {Find the total distance traveled on the close interval
      $\left[0,2\right]$}
    \begin{flalign*}
      \int\limits_{0}^{2}\sqrt{1+\left(\frac{-\sin t}{e^t}\right)^2}e^t dt \approx 6.558&&
    \end{flalign*}
     
  \item\textbf{Find the speed of the particle at $t=2$}
    \begin{flalign*}
      -\frac{\sin(t)}{e^t} \Rightarrow \frac{-\sin 2}{e^2} \approx -0.123&&
    \end{flalign*}
    
  \item\textbf{Find $\frac{dy}{dx}$}
    \begin{flalign*}
      -\frac{\sin (t)}{e^t}&&
    \end{flalign*}

  \item\textbf{Find $\frac{d^2 y}{dx^2}$}
    \begin{flalign*}
      \frac{\frac{d}{dt}\left[-\sin (t) e^{-t}\right]}{e^t} \Rightarrow
      \frac{-\left(\cos(t)e^{-t}+\left(-e^{-t}\right)\sin (t)\right)}{e^t} =
      -\frac{e^{-t}\left(\cos (t)-\sin (t)\right)}{e^t}&&
    \end{flalign*}

    \item\textbf{Let $p(t)$ be the distance, in meters, from the point $(0,1)$ at time
        $t$. $\vec{p}=\left\langle 2t,\cos(t)\right\rangle$}
      \begin{enumerate}[label=\alph*]
      \item
        \begin{flalign*}
          -\frac{\sin(t)}{2} \rightarrow \frac{-\frac{\cos(t)}{2}}{2} = -\frac{1}{4}\cos(t)&&
        \end{flalign*}
      \item
        \begin{flalign*}
          -\frac{1}{4}\cos \left(\cos\frac{5\pi}{6}\right) =
          -\frac{1}{4}\left(-\frac{\sqrt{3}}{2}\right) = \frac{\sqrt{3}}{8}&&
        \end{flalign*}

      \end{enumerate}
    \item\textbf{Using $\vec{p}=\left\langle 2t,\cos(t)\right\rangle$ explain what the
        acceleration of the particle means in relation to its position at
        $t=\frac{5\pi}{6}$}\\
      It means that at the time $\frac{5\pi}{6}$ the velocity of the particle is
      changing at a rate of $\frac{\sqrt{3}}{8}$
    \item\textbf {Determine the position vector given the following:}
     
      \begin{align*}
        \vec{a}(t)&=\left\langle \sqrt{t},t+1 \right\rangle\\
        \vec{v}(0)&=\left\langle 1 , 2 \right\rangle\\
        \vec{p}(0)&=\left\langle -1 , 5 \right\rangle
      \end{align*}
      \begin{flalign*}
        &\int \vec{a}(t)\ dt \rightarrow \int \vec{v}(t)\ dt \longrightarrow \vec{p}(t)& &&\\
        &\frac{2}{3}t^{\frac{3}{2}}+c \Rightarrow \frac{2}{3}0^{\frac{3}{2}}+c=1 \rightarrow c=1 & \frac{t^2}{2}+t+c \Rightarrow \frac{0^2}{2}+0+c=2 \rightarrow c=2 &&\\
        &\frac{2}{3}\cdot\frac{t^{\frac{5}{2}}}{\frac{5}{2}} + t + c \Rightarrow \frac{4}{15}0^{\frac{5}{2}}+ 0 +c =-1 \rightarrow c=-1 & \frac{t^3}{6}+\frac{t^2}{2}+2t+c=5 \Rightarrow  \frac{0^3}{6}+\frac{0^2}{2}+2(0)+c=5 \rightarrow c=5 && 
      \end{flalign*}
      \begin{align*}
        \vec{p}(t)=\left\langle \frac{4}{15}t^{\frac{5}{2}}+t-1\ ,\ \frac{t^3}{6}+\frac{t^2}{2}+2t+5 \right\rangle
      \end{align*}
      \textbf{Then use it to determine the value of $\vec{p}(7)$}\\
      \begin{flalign*}
        \vec{p}(7)=\left\langle \frac{4}{15}7^{\frac{5}{2}}+7-1\ ,\
          \frac{7^3}{6}+\frac{7^2}{2}+27+5 \right\rangle \Rightarrow
        \vec{p}(7)\approx \left\langle 40.571 , 100.667 \right\rangle&&
      \end{flalign*}
  \end{enumerate}
\end{document}